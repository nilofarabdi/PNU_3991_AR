{\documentclass [10pt,a4paper]{book}
	
\begin{document}
		
\begin{flushright}
			
WHAT IS THE NET ?      \textbf{19}
			
\end{flushright} 
these ideas are followed and created by the reader who also becomes author. A hyper text document, therefore, cannot be recreated on a conventional page of linear text.

 
In 1989, Berners-Lee, who was very familiar with Nelson's vision of hypertext, circu-lated a proposal that led to the development of the World Wide Web at CERN (the European Laboratory for Particle Physics). Bcrners-Lee's vision for hypertext saw a way to manage large amounts of information as follows: 
In providing a system for manipulating this sort of information, the hope would be to allow a pool of information to develop which could grow and evolve with the organiza-tion and the projects it describes. For this to be possible, the method of storage must not place its own restraints on the information. This is why a "Web" of notes with links (like references) between them is far more useful than a fixed hierarchical system. When describing u complex system, many people resort to diagrams with circles and arrows. Circles and arrows leave one free to describe the interrelationships between things in a way that tables, for example, do not. The system we need is like a diagram of circles and arrows, where circles and arrows can stand for anything. . ..The system must allow any sort of information to be entered. Another person must be able to find the information, sometimes without knowing what he (/she) is looking for. (Berners-Lee, 1989)

 
In October 1990, Bemers-Lee and Cailliau submitted a more detailed proposal to CERN for a "World Wide Web" hypertext project. A hypertext system, they pro-posed, would enable the linking of related data in physically separate locations. In sim-pler terms, hypertext allows for associations (links) between chunks of information (nodes). The subsequent system, referred to as a We!' (hence the name World Wide Web), was built to provide help manuals for users of the CERN computer systems. These basic elements, in combination with hypertext's other characteristics, allow for the production of extensive, flexible documents, especially when combined with mul-timedia (also referred to as hypermedia). 


Besides being described as the most complicated network, the Internet has also been described as the largest network ever constructed by human beings. This descrip-tion magnifies the extent and persuasiveness of the conununication capacity of the Net, but fails to give us a sense of what exactly the Internet is. Essentially, the Net's usefulness is built on two fundamental concepts. The first is an adherence to a single standard for communications between machines that are linked to the Internet. This standard, known by its technical name of Transportation Control Protocol/Internet Protocol (TCP/IP), is the Lingua Franca of all Net traffic. 

The TCP standards of the Internet protocol define how information is trans-ferred between hosts on the Internet. They detail how the information is broken into discrete packets, how these packets are transmitted, how the results are error-checked and retransmitted if errors are detected, and how the packets are reassembled at their final destination. As mentioned, this process takes place very quickly and is surprisingly robust, providing accurate transmission of error-free data. You can attest to this accu-racy yourself if you consider how rarely you receive an email from a friend that has any transmission-induced errors-99.99 percent of the time those typos were created by your colleague, not as a result of its transmission across the Net.
\end{document} 
