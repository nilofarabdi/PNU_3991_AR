\documentclass [12pt]{beamer}
\usepackage{xcolor}	
\usepackage{tikz}
\usetheme{Warsaw}
\useoutertheme{infolines}
\usepackage{ragged2e}
\begin{document}

\section*{kholase safahat 19...21}
\subsection*{nilufar abdi  }	
\begin{frame}
\justifying	
The storage method should not place its own restrictions on the data.  This is why "web" linked links (such as citations between them are much more useful than a fixed hierarchical system)
The Internet has been described as the largest man-made network.  This is a description of the extent and persuasiveness of pure communication capacity, but it cannot understand exactly what the Internet is
\end{frame}

\begin{frame}
\justifying	
A domain is a set of related URLs and web hosts that are controlled by an organization.  At the end of each address, at the root of the hierarchical system, lies the general, top-level domain name (educational institution).
When a user creates and sends an e-mail or requests from a website, a series of queries for the location of the address begins, and for each message packet a path is created between the source and destination
\end{frame}

\begin{frame}
\justifying	
When e-researchers have a good understanding of the network, they can use it more effectively in the research process.  In addition to understanding the Internet in both operational and technical ways, electronic researchers also need to understand how to use Internet search tools.  Academically, the main purpose of the Internet was to share information between scientists collaborating on research projects.
\end{frame}
\end{document}
